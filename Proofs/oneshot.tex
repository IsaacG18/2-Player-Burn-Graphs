\documentclass{article}
\usepackage{amsmath, amssymb, amsthm}

% Define the theorem environment for the proof
\newtheorem{theorem}{Theorem}
\newtheorem{lemma}{Lemma}
\newtheorem{definition}{Definition}

\begin{document}

% Title and author (optional)
\title{Transitivity of one shot 2 player burning graph on Trees}
\author{Isaac Gilbert}
\date{} % Remove date if not needed
\maketitle

\begin{definition}
A one shot 2 player burn graph denoted by \(B1\). Let G be a graph, \(G=(V,E)\). Let player 1 pick \(v\) and let player 2 pick \(u\) where \( v,u \in V \). If \(|\{ w \in V : d(v, w) < d(u, w) \}| > |\{ w \in V : d(u, w) < d(v, w) \}|\) then \(B1(G,v,u) =\) player 1 win, if \(|\{ w \in V : d(v, w) < d(u, w) \}| = |\{ w \in V : d(u, w) < d(v, w) \}|\) then \(B1(G,v,u) =\) player 1 ties, else if \(|\{ w \in V : d(v, w) < d(u, w) \}| < |\{ w \in V : d(u, w) < d(v, w) \}|\) then \(B1(G,v,u) =\) player 1 lose.
\end{definition}

\begin{definition}
A one shot 2 player burn graph game set denoted by \(B1S\). Let G be a graph, \(G=(V,E)\). Let player 1 pick \(v\) and let player 2 pick \(u\) where \( v,u \in V \). Then \(B1S(G,v,u) = \{ w \in V : d(v, w) < d(u, w) \}\)
\end{definition}

\begin{definition}
A one shot 2 player burn graph game count denoted by \(B1C\). Let G be a graph, \(G=(V,E)\). Let player 1 pick \(v\) and let player 2 pick \(u\) where \( v,u \in V \). Then \(B1S(G,v,u) = |\{ w \in V : d(v, w) < d(u, w) \}| - |\{ w \in V : d(u, w) < d(v, w) \}|\)
\end{definition}




\begin{lemma}
In a tree (a connected acyclic graph), there is exactly one path between any two vertices.
\end{lemma}

\begin{proof}
Suppose \( T \) is a tree, which by definition is a connected acyclic graph. We want to show that for any two vertices \( u \) and \( v \) in \( T \), there exists exactly one path between them.

1. **Existence of a Path**: Since \( T \) is connected, there is at least one path from \( u \) to \( v \).
   
2. **Uniqueness of the Path**: Assume, for contradiction, that there are two distinct paths \( P_1 \) and \( P_2 \) from \( u \) to \( v \). These two paths would form a cycle (since we can traverse from \( u \) to \( v \) along \( P_1 \) and return via \( P_2 \)), contradicting the fact that \( T \) is acyclic.

Thus, there is exactly one path between any two vertices in a tree.
\end{proof}

\begin{definition}
Let G be a tree, \(G=(V,E)\). Let \( v,u,w \in V \) then \(v\) is between \(w\) and \(u\) if the path from \(v\) to \(u\) does not contain \(w\) and  the path from \(v\) to \(w\) does not contain \(u\).
\end{definition}

\begin{definition}
Let G be a tree, \(G=(V,E)\). Let \( v,u,w \in V \) then \(v\) is behind \(w\) from \(u\) if the path from \(v\) to \(u\) contains \(w\).
\end{definition}


\begin{lemma}
In a graph G. Let Player 1 plays \( v_1 \), player 2 plays  \( v_2 \) for every vertex \( w \notin  \{v_1,v_2\} \) if every path in G from \( v_1 \to w\) include \( v_2 \) then \( w \in B1S(G,v_2,v_1)\) and  if every path in G from \( v_2 \to w\) include \( v_1 \) then \( w \in B1S(G,v_1,v_2)\).
\end{lemma}

\begin{proof}
Prove the first case. Using the definition of \(B1S\) we need to show  \(d(v_2, w) < d(v_1, w)\). Using the definition of distance which is the minimum number of edges in a path that connects two vertex and since every path for \( v_1 \to w\) include \( v_2 \) meaning this holds \(d(v_2, w) < d(v_1, w)\). The second case is true using the proof above.
\end{proof}

\begin{lemma}
In a tree G. Let Player 1 plays \( v_1 \), player 2 plays  \( v_2 \) then \( w \in  B1S(v_1, v_2) \) for all vertex if and only if \( v_1 \) is in the path \(v_2\) to \( w \)  or \( w \) is between \(v_2\) to \(v_1\), and \(d(v_1,w)<d(v_2,w)\).
\end{lemma}

\begin{proof}
First prove that if \( v_1 \) is in the path \(v_2\) to \( w \) then \( w \in  B1S(v_1, v_2) \). Using lemma 2 this is true.

Second prove that if  \( w \) is between \(v_2\) to \(v_1\) and \(d(v_1,w)<d(v_2,w)\) then \( w \in  B1S(v_1, v_2) \). Using definition of \(B1S(v_1, v_2) \), since \(d(v_1,w)<d(v_2,w)\) then \( w \in  B1S(v_1, v_2) \)

Third prove the only if. Then the only condition not account for is that \( v_2 \) is in the path \(v_1 \), \( w\) and the condition  \( w \) is between \(v_2\) to \(v_1\) and \(d(v_1,w)>=d(v_2,w)\). 

Since \( v_2 \) is in the path \(v_1, w\) then \( w \notin  B1S(v_1, v_2) \) using lemma 2 this is true then the first condition hold. Using definition of \(B1S(v_1, v_2) \), since \(d(v_1,w)>=d(v_2,w)\) then \( w \notin  B1S(v_1, v_2) \). Hence this is an if and only if.

\end{proof}



\begin{lemma}
In a tree G. If player 2 has not place adjacent to player 1 then player 2 increase its B1C for it to place on a vertex on its path between it current point and player 1 point.
\end{lemma}

\begin{proof}
Let \(G\) be a tree \( G=(V,E)\). Let player 1 pick vertex \(v_1 \) and player 2 pick vertex  \(v_2\) where \(d(v_1,v_2)>0\) 
\\ 
Let \(P\) be the path vertex on the path \( v_1 \) to \( v_2 \) with each vertex on the path denoted as \(p_1,...,p_n \) where \(n = d(v_1,v_2)\) If player 2 now select \( p_i\) where \( i \in [1,n]\). First we will prove that \(B1S(G,v_2, v_1) \subseteq B1S(G,p_i, v_1) \)
\\ \\
Using lemma 3 all points \(w\in B1S(G,v_2, v_1)\) must either have \( v_2 \) is in the path \(v_1\) to \( w \)  or \( w \) is between \(v_2\) to \(v_1\) and \(d(v_1,w)<d(v_2,w)\). 
\\ \\
Hence for any vertex \(w\) such that \( v_2 \) is in the path \(v_1\) to \( w \), since \(p_i\) must also be on that path then using Lemma 1 \(w \in B1S(G,p_i, v_1) \). 
\\ \\
For all vertex \( w \) is between \(v_2\) to \(v_1\) and \(d(v_1,w)<d(v_2,w)\). If \( w \) is not in the path to get to \(v_1\) or \(v_2\) and let \(p_j = min(d(w,p)) where p \in P\). Using lemma 1 \(p_j\) must be on both \(v_1\) and \(v_2\) to \(w\). If \(p_i\)  is on the path been \(v_1\) to \(p_j\) then using lemma 2 \(w \in B1S(G,p_i, v_1)\) . If \(p_j\) is on the path from \(v_1\) to \(p_i\) hence let \(x = d(v_2,p_j)\) hence \(y=d(v_1,p_j)\) and we have the following inequality \(x<y\) since it is in players 2 set. Since \(d(pi,pj) + d(pi,v2)=x\) then  \(d(pi,pj)<y\) hence \(w \in B1S(G,p_i, v_1)\). Since all vertex in \(B1S(G,v_2, v_1)\) are also in \( B1S(G,p_i, v_1) \) then \(B1S(G,v_2, v_1) \subseteq B1S(G,p_i, v_1) \)
\\ \\
Finally we need to find a vertex that is in \(B1S(G,v_1, v_2)\) that is not in \(B1S(G,v_1, p_i)\) or a vertex \(B1S(G,p_i, v_1)\) that is not in \(B1S(G,v_2, v_1)\). Let \(z\) be in \( P\) where \(d(v_2,z) - d(v_1,z) = 0\) when there is an odd number of vertex in \( P\) or \(d(v_2,z) - d(v_1,z) = 1\) when there is an even number of vertex in \( P\). If \(p_i\) is between \(v_1\) to \(z\) then \(z \in B1S(G,p_i, v_1)\) using lemma 2. If not then \(p_i\) is at least 1 close to hence \(z \in B1S(G,p_i, v_1)\) or \(z \in B1S(G,p_1, v_i)\) and therefore advantages for \(p_i\) since it is no worst and increase its own set or decrease the other players set.


\end{proof}

\begin{theorem}
Let G be a tree, then player 1 can always win or draw 1 shot burning graph.
\end{theorem}

\begin{proof}
Let player 1 select a vertex \(v\). For each adjacent vertex, count the total number of child nodes. If a vertex \(v'\) has more child nodes than all other adjacent vertices combined, designate \(v'\) as player 1's choice of vertex. Repeat this process until every adjacent vertex has fewer or an equal number of child nodes compared to the current choice, at which point player 1's selection process is complete.

Using Lemma 4, we know that the optimal strategy for player 2 is to select an adjacent vertex. Furthermore, by Lemma 1, we observe that every other child vertex becomes unreachable to player 2, as the only valid path leads through player 1’s selected vertex.

Thus, since each of player 1's child vertices has a count of nodes that is less than or equal to that of the other vertices, player 2’s choices are constrained to those with fewer or equal nodes. Consequently, player 2 can only achieve a tie or lose to player 1.
\end{proof}


\begin{theorem}
Let G be a tree, then if B1(G,v,u) is red win or draw and if B1(G,u,w) is red win or draw then B1(G,v,w) is red win or draw 
\end{theorem}

\begin{proof}

Case 1 let \(u\) be on the path \(v\) to \(w\): 
\\

Using Lemma 4 since \(u\) is on the path to \(v\) to \(w\) then \(u\) burns more than \(w\) against \(v\) since B1(G,v,u) is red win or draw then B1(G,v,w) is red win or draw
\\ \\
Case 2 let \(v\) be on the path \(u\) to \(w\):\\

Using Lemma 4 since \(v\) is on the path to \(u\) to \(w\) then \(v\) burns more than \(u\) against \(w\) since B1(G,u,w) is red win or draw then B1(G,v,w) is red win or draw
\\ \\
Case 3 let \(w\) be on the path \(v\) to \(u\):\\

Using Lemma 4 since \(w\) is on the path to \(v\) to \(u\) then \(w\) burns more than \(v\) against \(u\), however this is a contraction since B1(G,v,u) is red win or draw and if B1(G,u,w) which means \(w\) can not be on the path \(v\) to \(u\)
\\ \\
Case 4 let \(u\) to \(w\) with \(v\)  not being on the path, \(v\) to \(u\)  with \(w\)  not being on the path, and \(v\) to \(u\),  with \(u\)  not being on the path.
\\ \\
Let \(p\) a vertex that intersects the 3 paths.
\\\\
Lets split Case 4 up into cases
\\\\
Case 4.1: Let \(d(v,p) > d(u,p)\) hence select \(p\), then count the total number of child nodes that each of the connect nodes\(p\)has. 
Since\(u\)reaches\(p\)before\(v\)does then the total of the adjacent node that contains\(v\)is greater than any of the other combine. 
Hence for \(B1(G,u,w)\) to be true then \(d(u,p) <= d(w,p).\) Hence since\(w\)against\(v\)will burn the same or less of tree connect to\(p\)
that includes\(v\)then \(B1(G,v,w)\) is true since \(B1C(G,v,w) >= B1(G,v,u)\).
\\\\
Case 4.2: Let \(d(v,p) = d(u,p)\) hence select \(p\), then count the total number of child nodes 
that each of the connect nodes\(p\)has. Then the connect node that contains\(v\)must have a greater 
total the the connect node that contains \(u\). \(If d(u,p) < d(w,p)\) then the connect node that contains 
\(w\) is less than the rest of the connect nodes combine since \(B1(G,u,w)\) is true, since \(d(v,p) = d(u,p)\) 
then \(d(v,p) < d(w,p)\). If \(d(u,p) = d(w,p)\) then total number of child nodes of the connected node that 
containts\(v\)is greater than the total number of child noces of the connected node that containds\(u\)and 
the total number of child nodes of the connected node that containts\(u\)is greater than the total number of child 
noces of the connected node that containds\(w\)then total number of child nodes of the connected node that containts\(v\)
is greater than the total number of child noces of the connected node that containds w. Final since if \(d(u,p) > d(w,p)\) 
then\(w\)would reach\(p\)first and the total number of child nodes in the connected point that contains\(u\)is less than the 
rest combine since the total number of child nodes in the connected graph that contains\(v\)is greater than the total number 
of child nodes in the connect point that contains u. Hence this condition can not happen. 
\\\\
Case 4.3 Let d(v,p) < d(u,p) hence select p, then count the total number of child nodes 
that each of the connect nodes\(p\)has. If d(u,p) < d(w,p) then the connect node that contains 
w is less than the rest of the connect nodes combine since B1(G,u,w) is true, since d(v,p) < d(u,p) 
then d(v,p) < d(w,p). If d(u,p) = d(w,p) then the connect node that contains 
w is less than the connect node that contains\(u\)since B1(G,u,w) is true, since d(v,p) < d(u,p) 
then d(v,p) < d(w,p). Final since if d(u,p) > d(w,p) then\(w\)would reach\(p\)first and 
the total number of child nodes in the connected point that contains\(u\)is less than the 
rest combine since the total number of child nodes in the connected gdraph that contains\(v\)is greater than the total number 
of child nodes in the connect point that contains u. Hence this condition can not happen. 
\end{proof}


\end{document}